\section{Introduzione}

Questo corso si divide fondamentalmente in due parti:

\begin{enumerate}

	\item Una prima parte \textbf{teorica} in cui si studierà il software libero, le licenze software, il progetto GNU, ...
	\item Una seconda parte \textbf{pratica} che si baserà su diverse tecnologie e in cui verrà enfatizzato l'aspetto pratico.

\end{enumerate}

\subsection{Informazioni tecniche}

Sito web del corso:

\begin{center}

\url{http://www.math.unipd.it/~tapparo/TOS/}

\end{center}

Indirizzo email del docente:

\begin{center}

\url{tapparo@math.unipd.it} [\textit{attivo solo durante il periodo del corso}]

\end{center}

Le lezioni si terranno in \textbf{Aula 1BC50}

Gli orari sono i seguenti:

\begin{itemize}

\item \textbf{Giovedì}, 9:30 - 12:05
\item \textbf{Venerdì}, 9:30 - 12:05

\end{itemize}

Il ricevimento avverrà durante gli intervalli, su appuntamento e dopo lezione.

\textbf{48 ore} di lezione, \textbf{6 CFU}, tutte lezioni frontali.

La modalità d'esame è solamente \textbf{orale}, e avviene tramite iscrizione su \textit{Uniweb}. Verterà in due parti: la prima parte è una discussione sugli argomenti affrontati a lezione, la seconda è una parte pratica sulle tecnologie libere affrontate lungo il corso.

\subsection{Programma del corso}

\begin{itemize}

\item Storia del software libero;
\item Licenze libere e caratteristiche del software libero;
\item Problemi aperti e prospettive del software libero;
\item Strumenti liberi di supporto allo sviluppo e alla cooperazione.

\end{itemize}

\subsection{Materiale}

Appunti delle lezioni. [\textbf{PRINCIPALE}]

Materiale nelle \textbf{slides}.

Molti libri di riferimento si possono trovare nelle slides. Molti sono reperibili liberamente.

Libri:

\begin{itemize}

\item \textbf{Open Source: strategie, organizzazione}, è il più accademico e viene toccato marginalmente dal corso. Offre buoni spunti per quanto riguarda la gestione manageriale del software;
\item \textbf{Il software libero in Italia}, un libro molto interessante composto da diversi interventi. Contiene una buona sezione riguardante le licenze;
\item \textbf{Hackers: Heroes of the Computers}, libro leggibile come un romanzo, molto consigliato, molto leggero ma va ben oltre gli obiettivi del corso;
\item \textbf{Software libero, pensiero libero}, per chi ha poca dimestichezza con il progetto GNU. Anche questo libro è composto da una serie di interventi. \textit{Stallman} ha una grande capacità di organizzazione degli argomenti;
\item \textbf{Free culture}, di \textit{Lawrence Lessig}. Quest'ultimo, oltre a essere l'ex leader di \textit{Creative Common}, ha scritto una serie di libri in cui affronta le tematiche di libertà di accesso ai \textbf{contenuti}. È un libro scritto molto bene, affronta il problema della rivisitazione dei modelli di proprietà a fronte di forti cambiamenti (es. l'introduzione di Internet, l'introduzione dei voli aerei...);
\item \textbf{Privilege and property}, accessibile da Internet. Viene affrontata la nascita del copyright.

\end{itemize}

\subsection{Note su questi appunti}
Gli appunti sono stati realizzati in LaTeX e sono il prodotto dell'unione degli appunti presi a lezione e la trascrizione delle registrazioni nell'A.A. 2013/2014. 

Il contenuto degli appunti potrebbe non coprire eventuali aspetti ed argomenti tenuti negli anni accademici successivi, Il registro utilizzato è simile a quello tenuto a lezione.

Il PDF ottenuto, eventuali stampe e altre opere derivate da questo sorgente sono da intendersi come rilasciate sotto licenza CC-BY-SA 4.0\url{https://creativecommons.org/licenses/by-sa/4.0/}


\subsection{Introduzione al software libero}

Il software libero non ha nulla a che vedere con il \textit{prezzo}, ma è un software che rispetta \textbf{4 concetti fondamentali}, ovvero 4 libertà per l'utente:

\begin{itemize}

\item Libertà di \textbf{usare} il software, usandolo senza restrizioni. Es. libertà di prendere il prodotto ed utilizzarlo senza limiti di tempo, senza vincoli di paese e per \textit{qualunque scopo} (didattico, lavorativo, privato, ...). Il tipo di utilizzo non è mai precluso. Questa è da un lato la libertà meno importante, ma dal punto di vista dell'impatto sull'utente sviluppatore non è la maggiore;
\item Libertà di \textbf{studiare} il software. A differenza del software proprietario è possibile vedere il \textit{codice sorgente} e capire come funziona il software, ciò da una garanzia di protezione all'individuo. Senza questa libertà si ha un blocco della conoscenza ed è una forte limitazione alla crescita del prodotto;
\item Libertà di \textbf{modifica}, ovvero posso prendere il software e cambiarlo, costruire nuove soluzioni. Il software libero è visto in questo contesto come \textit{piattaforma}. Si tratta di costruire delle proprie versioni a partire da una base. È una libertà molto importante ed è una ricchezza per poi creare altri sviluppi;
\item Libertà di \textbf{ridistribuzione}, in questo caso le aziende non solo possono creare per se stesse, ma anche mettere la nuova versione a disposizione di altri. Software libero come bene comune (\glossario{Routes}, \glossario{Beowulf}, \glossario{nslu2}). Una volta che posso ridistribuire ad altri allora il mio lavoro diventa realmente usabile. La redistribuzione abbatte i costi e aumenta l'apporto di contributi tramite la community che acquisice competenze e vsibilità al migliorare del software. Con un piccolo sforzo di molti si ottiene un grande risultato.

\end{itemize}

\subsection{Libero != Gratuito}

È un errore comune confondere questi due concetti, ma essi sono realmente due cose distinte. Esiste software gratuito ma che non è libero ed esiste software libero non gratuito (\textit{openerp}, i programmi della \glossario{fsf}, i binari \glossario{RedHat Enterprise Linux}). C'è tutta una serie di software \glossario{freeware} o \glossario{shareware} (es. \textit{winzip}). Un software shareware è collegato ad un acquisto successivo, invita l'utente ad acquistare una versione commerciale.

\glossario{\textbf{openerp}} è un software a pagamento che fornisce supporto e assistenza tecnica garantendo plugin e funzionalità aggiuntive. 
\glossario{\textbf{Free Software Foundation}} distribuisce software libero disponibile per diverse architetture. Ha una serie di programmi non facili da compilare. Si scarica il sorgente e si tenta di compilarlo, oppure si richiede il CD con i file già compilati, e questo CD viene fatto pagare.

\glossario{\textbf{RedHat Enterprise}} risolve bug e problemi nel minor tempo possibile e fa il porting di nuove funzioni su vecchie versioni. Vengono distribuiti i sorgenti ma non i binari. Molte di queste modifiche apportate da sviluppatori RedHat vengono integrate in \glossario{\textbf{CentOS}}

I concetti di \textbf{libero} e \textbf{gratuito} sono dunque concetti ortogonali.

\subsection{L'importanza del software libero}

L'importanza del software libero è legata in primo luogo alla \textbf{riduzione dei costi} (apache, php, ...). Non è importante per il pagamento in sé ma in quanto mobilita le leggi del mercato. È un mercato aperto, con un tasso più alto di competizione ed innovazione nel quale è facile entrare (ha basse tariffe d'ingresso) ed investire.

Un secondo impatto riguarda la \textbf{trasparenza}. Se quello che faccio è visibile, è anche controllabile da altri sviluppatori. È difficile fidarsi di un software che non si sa bene cosa faccia. Il software libero è una \textbf{garanzia} in quanto il controllo collettivo migliora la qualità del software.

Con il software libero non abbiamo nessun \textbf{lock-in}, il software libero si può adattare facilmente alle versioni precedenti e quindi non si creano dipendenze da software specifico (come nel caso di software proprietario come \textit{Microsoft Office}).

\textbf{Sicurezza e affidabilità?} Non ci sono dimostrazioni effettive che questo sia vero. Da un lato il software libero è visibile a tutti, ma dall'altro la manutenzione è costosa ed è facile introdurre bug. Il software proprietario vive molto spesso di un inflazione di \textit{features}, questo per aumentare le vendite.

%TODO (citazione di Bill Gates)

Si passa da un modello gerarchico produttore - consumatore in cui c'è [\textbf{chi fa}] ed ha il potere derivato dalla conoscenza e [\textbf{chi usa}], e sta alle condizioni. Il software libero è conoscenza libera. Si cambia il modo in cui si sviluppa e con cui si fa impero. È un modello ``\textit{social}'', il software vale molto di più per il fatto che c'è una \textbf{comunità} che gli gira intorno. Il rapporto che si viene a creare con gli utenti è molto importante (\textit{Innovation happens elsewhere}). Intorno al software libero si può creare una comunità in modo che la somma dei costi per fare un prodotto è minore rispetto al costo della comunità stessa.

\subsection{Introduzione a GPL}

Per molto tempo il software libero ha avuto una \textbf{posizione di inferiorità}. Le aziende prendevano software libero, sviluppavano una nuova versione e le rilasciavano come software proprietario. Il progetto \textbf{GNU} voleva creare una versione completamente libera del software. Ha creato dunque una nuova licenza, chiamata \textbf{GPL} (General Public License), in modo che avesse un \textit{effetto virale}. Una licenza libera ma con una restrizione particolare, ovvero ogni redistribuzione deve essere rilasciata sotto licenza GPL (circolo virtuoso e virale). Vedere il software libero come un'enorme libreria di conoscenza sempre disponibile.

Il software libero con questa licenza si arricchisce molto, cresce nel tempo e diventa sempre più interessante. Questa licenza è ancora molto presente (60\%, 70\% del software libero è sotto GPL).