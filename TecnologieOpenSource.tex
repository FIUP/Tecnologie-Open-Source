%%%%%%%%%%%%%%%%%%%%%%%%%%%%%%%%%%%%%%%%%
% Classic Lined Title Page 
% LaTeX Template
% Version 1.0 (27/12/12)
%
% This template has been downloaded from:
% http://www.LaTeXTemplates.com
%
% Original author:
% Peter Wilson (herries.press@earthlink.net)
%
% License:
% CC BY-NC-SA 3.0 (http://creativecommons.org/licenses/by-nc-sa/3.0/)
% 
% Instructions for using this template:
% This title page compiles as is. If you wish to include this title page in 
% another document, you will need to copy everything before 
% \begin{document} into the preamble of your document. The title page is
% then included using \titleAT within your document.
%
%%%%%%%%%%%%%%%%%%%%%%%%%%%%%%%%%%%%%%%%%

%----------------------------------------------------------------------------------------
%	PACKAGES AND OTHER DOCUMENT CONFIGURATIONS
%----------------------------------------------------------------------------------------



\documentclass[a4paper]{article}

\usepackage[utf8]{inputenc}
\usepackage{hyperref}
\usepackage{titlesec}
\usepackage{amssymb,amsmath}
\usepackage[italian]{babel}
\usepackage[titles]{tocloft}
\usepackage{appendix}
\hypersetup{%
    pdfborder = {0 0 0}
}
\usepackage{graphicx}
\usepackage[svgnames]{xcolor} % Required to specify font color
\usepackage{eurosym}

\newcommand*{\plogo}{\fbox{$\mathcal{PL}$}} % Generic publisher logo

%----------------------------------------------------------------------------------------
%	TITLE PAGE
%----------------------------------------------------------------------------------------

\newcommand*{\titleAT}{\begingroup % Create the command for including the title page in the document
\newlength{\drop} % Command for generating a specific amount of whitespace
\drop=0.1\textheight % Define the command as 10% of the total text height

\rule{\textwidth}{1pt}\par % Thick horizontal line
\vspace{2pt}\vspace{-\baselineskip} % Whitespace between lines
\rule{\textwidth}{0.4pt}\par % Thin horizontal line

\vspace{\drop} % Whitespace between the top lines and title
\centering % Center all text
\textcolor{Black}{ % Red font color
{\Huge Appunti}\\[0.5\baselineskip] % Title line 1
{\Large di}\\[0.75\baselineskip] % Title line 2
{\Huge Tecnologie Open Source}} % Title line 3

\vspace{0.25\drop} % Whitespace between the title and short horizontal line
\rule{0.3\textwidth}{0.4pt}\par % Short horizontal line under the title
\vspace{\drop} % Whitespace between the thin horizontal line and the author name

{\Large Luca De Franceschi}\par % Author name

\vfill % Whitespace between the author name and publisher text
{\large Università degli studi di Padova}\par % Publisher

\vspace*{\drop} % Whitespace under the publisher text

\rule{\textwidth}{0.4pt}\par % Thin horizontal line
\vspace{2pt}\vspace{-\baselineskip} % Whitespace between lines
\rule{\textwidth}{1pt}\par % Thick horizontal line

\endgroup}

\titleformat{\chapter}[display]
{}{\hfill\rule{.7\textwidth}{3pt}}{2pt}
{\hspace*{.3\textwidth}\huge\bfseries}[\addvspace{1pt}]
\titleformat{name=\chapter,numberless}[display]
{}{\hfill\rule{.7\textwidth}{3pt}}{2pt}
{\hspace*{.3\textwidth}\huge\bfseries}[\addvspace{1pt}]

\renewcommand*\contentsname{Indice}

\newcommand{\glossario}[1]{\textit{#1\ped{G}}}

%----------------------------------------------------------------------------------------
%	BLANK DOCUMENT
%----------------------------------------------------------------------------------------

\begin{document}
\raggedright
\titleAT % This command includes the title page
\thispagestyle{empty}
\newpage

\pagenumbering{arabic}

\tableofcontents
\clearpage

\section{Introduzione}

Questo corso si divide fondamentalmente in due parti:

\begin{enumerate}

	\item Una prima parte \textbf{teorica} in cui si studierà il software libero, le licenze software, il progetto GNU, ...
	\item Una seconda parte \textbf{pratica} che si baserà su diverse tecnologie e in cui verrà enfatizzato l'aspetto pratico.

\end{enumerate}

\subsection{Informazioni tecniche}

Sito web del corso:

\begin{center}

\url{http://www.math.unipd.it/~tapparo/TOS/}

\end{center}

Indirizzo email del docente:

\begin{center}

\url{tapparo@math.unipd.it} [\textit{attivo solo durante il periodo del corso}]

\end{center}

Le lezioni si terranno in \textbf{Aula 1BC50}

Gli orari sono i seguenti:

\begin{itemize}

\item \textbf{Giovedì}, 9:30 - 12:05
\item \textbf{Venerdì}, 9:30 - 12:05

\end{itemize}

Il ricevimento avverrà durante gli intervalli, su appuntamento e dopo lezione.

\textbf{48 ore} di lezione, \textbf{6 CFU}, tutte lezioni frontali.

La modalità d'esame è solamente \textbf{orale}, e avviene tramite iscrizione su \textit{Uniweb}. Verterà in due parti: la prima parte è una discussione sugli argomenti affrontati a lezione, la seconda è una parte pratica sulle tecnologie libere affrontate lungo il corso.

\subsection{Programma del corso}

\begin{itemize}

\item Storia del software libero;
\item Licenze libere e caratteristiche del software libero;
\item Problemi aperti e prospettive del software libero;
\item Strumenti liberi di supporto allo sviluppo e alla cooperazione.

\end{itemize}

\subsection{Materiale}

Appunti delle lezioni. [\textbf{PRINCIPALE}]

Materiale nelle \textbf{slides}.

Molti libri di riferimento si possono trovare nelle slides. Molti sono reperibili liberamente.

Libri:

\begin{itemize}

\item \textbf{Open Source: strategie, organizzazione}, è il più accademico e viene toccato marginalmente dal corso. Offre buoni spunti per quanto riguarda la gestione manageriale del software;
\item \textbf{Il software libero in Italia}, un libro molto interessante composto da diversi interventi. Contiene una buona sezione riguardante le licenze;
\item \textbf{Hackers: Heroes of the Computers}, libro leggibile come un romanzo, molto consigliato, molto leggero ma va ben oltre gli obiettivi del corso;
\item \textbf{Software libero, pensiero libero}, per chi ha poca dimestichezza con il progetto GNU. Anche questo libro è composto da una serie di interventi. \textit{Stallman} ha una grande capacità di organizzazione degli argomenti;
\item \textbf{Free culture}, di \textit{Lawrence Lessig}. Quest'ultimo, oltre a essere l'ex leader di \textit{Creative Common}, ha scritto una serie di libri in cui affronta le tematiche di libertà di accesso ai \textbf{contenuti}. È un libro scritto molto bene, affronta il problema della rivisitazione dei modelli di proprietà a fronte di forti cambiamenti (es. l'introduzione di Internet, l'introduzione dei voli aerei...);
\item \textbf{Privilege and property}, accessibile da Internet. Viene affrontata la nascita del copyright.

\end{itemize}

\subsection{Note su questi appunti}
Gli appunti sono stati realizzati in LaTeX e sono il prodotto dell'unione degli appunti presi a lezione e la trascrizione delle registrazioni nell'A.A. 2013/2014. 

Il contenuto degli appunti potrebbe non coprire eventuali aspetti ed argomenti tenuti negli anni accademici successivi, Il registro utilizzato è simile a quello tenuto a lezione.

Il PDF ottenuto, eventuali stampe e altre opere derivate da questo sorgente sono da intendersi come rilasciate sotto licenza CC-BY-SA 4.0\url{https://creativecommons.org/licenses/by-sa/4.0/}


\subsection{Introduzione al software libero}

Il software libero non ha nulla a che vedere con il \textit{prezzo}, ma è un software che rispetta \textbf{4 concetti fondamentali}, ovvero 4 libertà per l'utente:

\begin{itemize}

\item Libertà di \textbf{usare} il software, usandolo senza restrizioni. Es. libertà di prendere il prodotto ed utilizzarlo senza limiti di tempo, senza vincoli di paese e per \textit{qualunque scopo} (didattico, lavorativo, privato, ...). Il tipo di utilizzo non è mai precluso. Questa è da un lato la libertà meno importante, ma dal punto di vista dell'impatto sull'utente sviluppatore non è la maggiore;
\item Libertà di \textbf{studiare} il software. A differenza del software proprietario è possibile vedere il \textit{codice sorgente} e capire come funziona il software, ciò da una garanzia di protezione all'individuo. Senza questa libertà si ha un blocco della conoscenza ed è una forte limitazione alla crescita del prodotto;
\item Libertà di \textbf{modifica}, ovvero posso prendere il software e cambiarlo, costruire nuove soluzioni. Il software libero è visto in questo contesto come \textit{piattaforma}. Si tratta di costruire delle proprie versioni a partire da una base. È una libertà molto importante ed è una ricchezza per poi creare altri sviluppi;
\item Libertà di \textbf{ridistribuzione}, in questo caso le aziende non solo possono creare per se stesse, ma anche mettere la nuova versione a disposizione di altri. Software libero come bene comune (\glossario{Routes}, \glossario{Beowulf}, \glossario{nslu2}). Una volta che posso ridistribuire ad altri allora il mio lavoro diventa realmente usabile. La redistribuzione abbatte i costi e aumenta l'apporto di contributi tramite la community che acquisice competenze e vsibilità al migliorare del software. Con un piccolo sforzo di molti si ottiene un grande risultato.

\end{itemize}

\subsection{Libero != Gratuito}

È un errore comune confondere questi due concetti, ma essi sono realmente due cose distinte. Esiste software gratuito ma che non è libero ed esiste software libero non gratuito (\textit{openerp}, i programmi della \glossario{fsf}, i binari \glossario{RedHat Enterprise Linux}). C'è tutta una serie di software \glossario{freeware} o \glossario{shareware} (es. \textit{winzip}). Un software shareware è collegato ad un acquisto successivo, invita l'utente ad acquistare una versione commerciale.

\glossario{\textbf{openerp}} è un software a pagamento che fornisce supporto e assistenza tecnica garantendo plugin e funzionalità aggiuntive. 
\glossario{\textbf{Free Software Foundation}} distribuisce software libero disponibile per diverse architetture. Ha una serie di programmi non facili da compilare. Si scarica il sorgente e si tenta di compilarlo, oppure si richiede il CD con i file già compilati, e questo CD viene fatto pagare.

\glossario{\textbf{RedHat Enterprise}} risolve bug e problemi nel minor tempo possibile e fa il porting di nuove funzioni su vecchie versioni. Vengono distribuiti i sorgenti ma non i binari. Molte di queste modifiche apportate da sviluppatori RedHat vengono integrate in \glossario{\textbf{CentOS}}

I concetti di \textbf{libero} e \textbf{gratuito} sono dunque concetti ortogonali.

\subsection{L'importanza del software libero}

L'importanza del software libero è legata in primo luogo alla \textbf{riduzione dei costi} (apache, php, ...). Non è importante per il pagamento in sé ma in quanto mobilita le leggi del mercato. È un mercato aperto, con un tasso più alto di competizione ed innovazione nel quale è facile entrare (ha basse tariffe d'ingresso) ed investire.

Un secondo impatto riguarda la \textbf{trasparenza}. Se quello che faccio è visibile, è anche controllabile da altri sviluppatori. È difficile fidarsi di un software che non si sa bene cosa faccia. Il software libero è una \textbf{garanzia} in quanto il controllo collettivo migliora la qualità del software.

Con il software libero non abbiamo nessun \textbf{lock-in}, il software libero si può adattare facilmente alle versioni precedenti e quindi non si creano dipendenze da software specifico (come nel caso di software proprietario come \textit{Microsoft Office}).

\textbf{Sicurezza e affidabilità?} Non ci sono dimostrazioni effettive che questo sia vero. Da un lato il software libero è visibile a tutti, ma dall'altro la manutenzione è costosa ed è facile introdurre bug. Il software proprietario vive molto spesso di un inflazione di \textit{features}, questo per aumentare le vendite.

%TODO (citazione di Bill Gates)

Si passa da un modello gerarchico produttore - consumatore in cui c'è [\textbf{chi fa}] ed ha il potere derivato dalla conoscenza e [\textbf{chi usa}], e sta alle condizioni. Il software libero è conoscenza libera. Si cambia il modo in cui si sviluppa e con cui si fa impero. È un modello ``\textit{social}'', il software vale molto di più per il fatto che c'è una \textbf{comunità} che gli gira intorno. Il rapporto che si viene a creare con gli utenti è molto importante (\textit{Innovation happens elsewhere}). Intorno al software libero si può creare una comunità in modo che la somma dei costi per fare un prodotto è minore rispetto al costo della comunità stessa.

\subsection{Introduzione a GPL}

Per molto tempo il software libero ha avuto una \textbf{posizione di inferiorità}. Le aziende prendevano software libero, sviluppavano una nuova versione e le rilasciavano come software proprietario. Il progetto \textbf{GNU} voleva creare una versione completamente libera del software. Ha creato dunque una nuova licenza, chiamata \textbf{GPL} (General Public License), in modo che avesse un \textit{effetto virale}. Una licenza libera ma con una restrizione particolare, ovvero ogni redistribuzione deve essere rilasciata sotto licenza GPL (circolo virtuoso e virale). Vedere il software libero come un'enorme libreria di conoscenza sempre disponibile.

Il software libero con questa licenza si arricchisce molto, cresce nel tempo e diventa sempre più interessante. Questa licenza è ancora molto presente (60\%, 70\% del software libero è sotto GPL).
\clearpage
\section{La nascita del copyright}

\subsection*{Materiale di riferimento}

\begin{itemize}

\item \textit{Privilege and Property Essays on the History of Copyright} - Ronan Deazley, OpenBook Publishers;
\item \url{http://digital-law-online.info/lpdi1.0/index.html} - sezione riguardo il copyright sul software;

\end{itemize}

\subsection{Le origini}

Le prime forme di protezione in realtà non erano pensate per proteggere i diritti del beneficiario ma per dare dei vantaggi alle autorità che le emanavano; in secondo luogo non erano collegate alla conoscenza che raccoglieva quello si voleva proteggere, non si proteggeva il contenuto del libro ma si proteggeva l'ente industriale che lo aveva prodotto. Infine non erano collegate nemmeno all'autore. La forma era molto diversa da quella odierna.

Il copyright si è sviluppato in origine a Venezia intorno al 1469, 13 anni dopo la produzione della bibbia di Gutemberg. Prima dell'invenzione della stampa non c'era un sistema ben strutturato e organizzato, il libri costavano moltissimo, richiedeva anni di lavoro e di conseguenze non ve ne erano molti. Era un processo molto impegnativo la scrittura di un libro, stimato intorno agli 80.000 \euro{} di oggi. Nel 1450 la bibbia di Gutemberg cambia decisamente le carte in gioco, viene creato un processo industriale della scrittura, che prima era quasi un'``opera d'arte''. Lo stato incominciò ad interessarsene, per controllare il flusso di informazioni ed imporre dei blocchi sulla conoscenza; questa fu la direzione presa in Inghilterra. Dall'altro lato la stampa era un'invenzione fenomenale e si voleva trarre vantaggi da essa; questa fu la direzione presa a Venezia, i quali erano molto interessati ad avere il sistema di stampa e a sfruttarlo; cercarono di fare in modo che tanta gente ce l'avesse, imponendo comunque dei controlli. In quest'ottica il copyright non nasce come un diritto, ma come forma di privilegio che l'autorità concede, ha una forma di incentivo brevettuale.

In Italia esistevano delle \textbf{corporazioni}, che detenevano il controllo sulla conoscenza delle arti artigiane, vi era tutto un sistema di privilegi ed erano loro a mantenere l'ordine. Dall'altra parte gli stessi comuni che avevano creato queste corporazioni avevano anche creato un sistema per incentivare la gente degli altri comuni di svelare la conoscenza e diffondere le tecniche più avanzate. 

Quando nel 1469 Johannes of Speyer andò a Venezia chiese una forma di incentivo per portare la propria macchina di stampa a Venezia, e ovviamente glielo concessero, perchè era una macchina importante dalle grandi potenzialità. Gli diedero dunque un'esclusiva sulla stampa per 5 anni. Al giorno d'oggi quello che conta di un libro è il suo contenuto, non la forma; all'epoca si pagavano i libri in funzione del peso o come merce di scambio. Pochi mesi dopo questa esclusiva però Johannes morì, e questo privilegio durò dunque per pochi mesi, e preso si cominciò dunque a formare un mercato sulla stampa. 

Una cosa importante che fu conseguenza di questo privilegio fu che il controllo della stampa venne sottratto alle corporazioni, la produzione si orientò dunque in un certo modo, non ci fu la differenziazione del modo in cui venivano gestiti i privilegi di stampa. Questi privilegi erano concessi volta per volta alle singole persone ed erano associati al modo in cui venivano prodotti i libri:

\begin{itemize}

\item Privilegi e non diritti d'autore; 
\item Era lo stampatore ad avere i diritti;
\item Carattere tecnologiche dei privilegi iniziali, non associati al contenuto.

\end{itemize}

Le opere all'epoca erano per la maggior parte diverse edizioni delle opere classiche.

Uno \textbf{statuto} importantissimo, quello del 1474, per la prima volta stabiliva che quando una persona produceva qualche cosa di meritevole, di nuovo e originale, aveva diritto ad una protezione per 10 anni. Questa sembra per la prima volta una forma di protezione collegata alla proprietà intellettuale di quello che ci sta dentro e non esclusivamente ad un processo industriale. Era una cosa che si avvicinava a un \textit{diritto}. Ma di fatto questo statuto finì in un binario morto ma ebbe un effetto molto importante, perchè si spostò l'attenzione per la prima volta dall'interesse degli stampatori agli \textbf{autori}. 

\subsection{Il rinascimento}

Il 1517 segna un cambiamento nel modo con cui vengono distribuiti i privilegi. Prima i privilegi venivano distribuiti in modo indipendente dal contenuto, quello che interessava era esclusivamente il processo di stampa. A un certo punto quando ci si stanca di avere tutti libri uguali della stessa opera, volevano avere libri un po' più \textit{nuovi}, e quindi ritirarono tutti i privilegi sui libri in stampa e le opere cadono nel pubblico dominio. Fu necessario dunque lo spostamento del mercato verso le opere originali, le quali erano proteggibili. Per una volta quello che conta non è il modo in cui viene stampato un libro ma quello che ci sta dentro. Gli \textbf{autori} cominciarono ad avere dunque un po' più di potere. Questa protezione sulle opere si rafforzò nel tempo, andando a proteggere anche le \textbf{modifiche} sulle opere. Come conseguenza di questo i privilegi cominciarono ad essere garantiti anche agli autori.  

La regolamentazione delle arti artigiane era fatta dalle corporazioni, ma piano piano sempre più persone le stavano trovando più adatte. Si stava entrando nel rinascimento, un'epoca in cui si da più spazio all'uomo e alla sua creatività. Le corporazioni avevano il compito di regolamentare le arti, di proteggerle; ogni comune aveva le proprie. Una conseguenza importante di ciò fu la nascita del concetto di ``\textbf{proprietà immateriale}''. Si aveva la netta sensazione che quello che una persona conosceva era importante. D'altra parte la forma di protezione era strettamente legata alla comunità e la comunità va protetta proteggendo l'informazione. Questa forma di privilegio era molto legata agli autori e non più ai produttori. Si spostò l'interesse dal processo di produzione del libro al suo contenuto e in particolare all'autore.

Il sistema brevettuale parallelamente collegò il concetto di proprietà immateriale alla persona, anche perchè questo è un periodo in cui c'è uno spostamento generale dalla comunità alla persona (umanesimo). Questo ebbe un grosso impatto. Prima chi gestiva la conoscenza erano degli artigiani, con la nascita di un interesse culturale diventa più ``teorico'', nasce una differenza tra la proprietà intellettuale e i suoi prodotti. Questo venne rafforzato ulteriormente dalla nascita degli \textbf{scrittori di professioni}. Il valore delle opere deriva dall'individuo e dalle sue conoscenze.

Nel 1600 in Inghilterra si era sviluppata tutta una vita politica pubblica, in cui si discuteva pubblicamente e ci si formava un'opinione, per esempio nei vari caffè. A un certo punto si sentì il bisogno di controllare questa opinione pubblica; prima esisteva una struttura chiamata \textit{camera stellata} che era un tribunale ``fittizio'', una forma di censura che permetteva di controllare ciò che veniva espresso dalla gente. Nel 1641 venne abolita e a quel punto ci fu la necessità di sostituire questa forma censoria. Questi controlli vengono implementati nel 1643/1644, in ci ci fu un rigido controllo pre-stampa e la censura fu affidata alla \textbf{stationer's company}, alla quale era affidato il compito di decidere chi poteva stampare. Naturalmente il diritto lo aveva solo chi si dimostrava premuroso nei confronti dei diritti della corona. Questa legge venne prorogata diverse volte fino al 1695. 

Ci furono tutta una serie di tentativi (13) di restaurazione dei controlli censori, cercarono di ripristinare la legge, ma gli editti erano cambiati, quindi di fatto non ci riuscirono mai. Alla fine o si lasciava la stampa completamente libera oppure si trovava un altro modo di ripristinare una forma di controllo che fosse abbastanza contenibile per i diritti di allora. La soluzione fu di dare una licenza agli autori, dare loro una certa libertà, con l'\textbf{editto di Ann} del 1710. Il monopolio dell'autore aveva una durata di 14 anni, con la possibilità di chiederne ulteriori 14. Questo è il primo vero esempio di copyright. 

\subsection{Il nuovo mondo}

In America invece ci fu un approccio un po' misto tra quello sviluppatosi in Inghilterra e a Venezia. L'America era comunque una colonia inglese quindi risentiva in modo molto forte delle censure dell'Inghilterra. Nel 1638 il reverendo Glover porta la prima macchina a stampa in Massachussets, con lo scopo di divulgare il vangelo. Vi era un mix di controllo e di patrocinio.  Il primo privilegio di stampa è del 1672, si offriva di stampare qualcosa che fosse nell'interesse della comunità, e in cambio si chiedeva un aiuto. Questo fu un esempio di privilegio molto simile a quello di Venezia. 

Andrew Law fu il primo ad ottenere l'accordo sul privilegio d'autore nel 1781. Aveva paura che il suo stampatore gli fregasse il lavoro, quindi chiese ed ottenne questo privilegio legato al contenuto dell'opera. Si arriva ad una vera e propria forma di copyright nel 1783, in cui John Ledyard aveva chiesto di avere un privilegio di stampa; la cosa nuova e inaspettata fu che chi doveva decidere se concedere o meno questo privilegio doveva attenersi a delle regole, quindi venne creato il \textbf{Connecticut copyright statute}. Nel 1790 questo decreto venne poi trasformato in un \textbf{Copyright act} che valeva in senso generale.

\subsection{Il copyright moderno}

Nel 1883 ci fu la \textbf{convenzione di Berna} che stabilì per la prima volta una regolamentazione internazionale, ragione per cui adesso è possibile parlare di copyright in senso generale. Ha subito una serie di modifiche nel tempo, fino al 1979, ma la grossa modifica fu che la tutela divenne automatica, senza registrazione. Attualmente include 165 paesi. 

\subsection{Il copyright sul software}

Finora abbiamo parlato di copyright relativo alla carta stampata. Quando si parla di copyright sul \textbf{software} le cose cambiano drasticamente, perchè non è molto chiaro che ciò che risiede sulla memoria di un computer possa essere proteggibile dal copyright. Oggi noi lo diamo per scontato ma libro e software hanno proprietà molto diverse. Il software non è una cosa tangibile, è qualcosa di ``nascosto''. L'idea fondamentale della legge sul copyright è che esso serve essenzialmente per proteggere la comunità, in modo da incentivare l'autore a condividere il proprio lavoro. Nel 1950 nascono i primi computer. Da lì non ci fu la necessità immediata di proteggere il software, perchè di computer ce n'erano molto pochi, costavano molto e i lavori erano commissionati. Il codice sorgente una volta utilizzato veniva ``buttato via''. Si inizia a pensare di proteggere il software nel 1964, in cui non c'è protezione da parte di un'azienda ma c'era un gruppo di studenti che volevano vedere se fosse possibile proteggere il software. Nel 1976, con il copyright act, risultò chiaro che il software era proteggibile. 

\subsection{Il diritto d'autore in Italia}

Diritto esclusivo dell'autore su:

\begin{itemize}

\item Ridistribuzione;
\item Modifica;
\item Adattamento;
\item Traduzione.

\end{itemize}

Tale diritto è \textbf{rinunciabile} e \textbf{trasferibile}.

\clearpage
\section{Storia del software libero}

\subsection{Gli albori}

1950 - 1960, la cultura hacker nasce ai laboratori del \textbf{MIT}. Nel 58/59 nascono i primi corsi di \textbf{intelligenza artificiale} (di fatto era informatica). Vi era un rapporto \textit{giocoso} tra il gruppo di ricerca e i ragazzi, un rapporto che funzionava molto bene. Il primo gruppo hacker nasce come sottogruppo di \glossario{S\&P}. All'inizio non era una vera e propria comunità hacker ma c'era solamente un gruppo che frequentava i corsi e morta lì. L'accesso ai PC all'epoca era molto riservato (ai docenti, al personale). Questo gruppo aveva trovato una sala (407) con delle macchine perforatrici (per programmare le schede perforate). 

Il rapporto cambierà con l'evoluzione delle tecnologie, con l'accesso più libero alle nuove macchine (ATX0). Con il cambiamento delle tecnologie cambia anche la gestione (si poteva accedere liberamente alle macchine), si cominciò a formare un gruppo di persone che \textit{bazzicavano} sulle macchine. Si formò così una comunità con idee e un'etica in comune. 

Con il \glossario{PDP} (computer) le cose cambiarono ulteriormente. IL PDP era pensato non per il \textit{best-processing} ma per una computazione interattiva (aveva un monitor ...), costava inoltre molto meno ed era quindi più accessibile. A quel punto diventava possibile utilizzare le macchine a costo zero.

\subsection{Etica hacker}

L'etica che si sviluppò all'interno della comunità si basava sui seguenti 4 punti:

\begin{enumerate}

\item \textbf{Libero accesso all'informazione}, bisogna poter metterci le mani, e questo non era all'epoca garantito a tutti;
\item \textbf{Decentralizzazione}, potere che si sposta dal centro;
\item Gli hacker dovrebbero essere \textbf{giudicati solo per il loro valore};
\item Il software come \textbf{espressione artistica}, va oltre quello che è la pura utilità, va a fare qualcosa che è divertente e bello (creazione di giochi). Doveva essere un piacere che andasse agli altri;

\end{enumerate}

\subsection{Conseguenze dello sviluppo software}

In questo momento c'è una libera condivisione del codice, non esiste software protetto da copyright. Non c'era interesse nel proteggerlo. Il software era un'appendice dell'hardware. Le macchine erano in continua evoluzione e modifica. L'avere un accesso a come il software funzionava era vitale per gli sviluppatori. Dato che il tempo macchina era oneroso, bisognava sfruttarlo al meglio e quindi era importante la condivisione del lavoro tra gli sviluppatori.

\textbf{Incopatible Time Sharing}, ci si compra un terminale e poi si accede alle reti di calcolo di un enorme computer. ITS era la base di questo concetto. Non c'erano password, ogni utente aveva i propri file personali accedibili da tutti. Era disegnato per la cooperazione tra gli hacker ed assumeva una grossa fiducia negli utenti.

\subsection{La fine del periodo degli hacker}

\begin{itemize}

\item 68: il computer lab viene isolato;
\item 70's-80's: frammentazione della cultura hacker:

	\begin{itemize}

	\item 1976: copyright act;
	\item \glossario{Symbolics} recluta la maggior parte degli hacker rimasti al MIT;
	\item Altri hacker fondano \glossario{LMI}.

	\end{itemize}

\item Negli '80 cessa la produzione del PDP-10, condannando definitivamente ITS.

\end{itemize}

\subsection{Nasce unix}

Una serie di circostanze ha voluto che \glossario{unix} venisse rilasciato con licenza libera (era comunque protetto da copyright). Nasce come una risposta a \glossario{MUTIX} per creare un sistema diverso, più piccolo e con minor risorse. Fu il primo sistema scritto non direttamente in linguaggio macchina ma scritto in linguaggio C. Questo sistema ebbe subito un grosso successo anche e soprattutto in ambito universitario. 

L'\glossario{AT\&T} era un monopolio telefonico del governo, aveva la restrizione che non poteva vendere software e quindi si decise di distribuire gratuitamente unix. Al suo lancio esso si diffuse molto rapidamente. Una delle università che lo utilizzò fin da subito fu \textbf{Berkeley}, che aveva interesse dal punto di vista della programmazione interna di unix. Nel 1977 nacque \glossario{BSD}, una versione modificata di unix utilizzata dall'università. Due anni dopo la AT\&T annunciò di voler commercializzare unix. 

Nel 1983 AT\&T viene divisa e unix diventa commerciale. A questo punto c'era BSD ed era una possibile scelta. \glossario{ARPA} lo utilizzò per creare \glossario{ARPANET}, per collegare diversi computer via network, garantendo un'uniformità a livello software del sistema operativo. Il software libero ebbe quindi un impatto notevole. ARPA mise inoltre a disposizione dei fondi per migliorare BSD e facilitare il suo sviluppo. In seguito venne citata in giudizio dalla AT\&T perchè si diceva volesse il codice di unix. Questo bloccò di fatto lo sviluppo di ARPANET. Se ciò non fosse successo probabilmente non sarebbe nato unix.

\subsection{Nasce il progetto GNU}


\clearpage
\section{L'ultimo degli hacker}

Analizziamo la fase di decadenza degli hacker, qui entra in gioco Richard Stallman.

\subsection{Prime esperienze}

Nasce a New York dove si forma dal punto di vista informatico, diventa un esperto di \glossario{Assembler}, di sistemi operativi e di editor di testo. Poi entra ad Harvard e si laurea in Fisica. Scopre in segreto un'affinità con la cultura hacker sviluppatasi al MIT. Venne assunto al MIT come programmatore di sistemi. Viene preso sotto l'ala protettiva di Richard Greenblatt e Bill Gosper, che gli fanno da mentore.

\subsection{Emacs}

Una delle prime cose che Stallman fece fu la creazione di \glossario{Emacs}. Inizialmente esso era un insieme di macro per \glossario{TECO}, che era una sorta di linguaggio per scrivere e non era assolutamente real-time. Creò dunque una sorta di editor dentro TECO. Guy Steele ebbe in seguito l'idea di fare un po' di ordine tra le macro (che erano diventate tantissime). Quest'opera fu continuata da Stallman, in modo da avere un insieme più omogeneo. Questo disordine non sarebbe dovuto replicarsi in futuro, bisognava evitare che ciascuno costruisse il proprio insieme di macro. Quindi si imposero delle restrizioni sulla redistribuzione delle macro. Questa condizione creò una sorta di comunità in cui i programmatori condividevano gli sforzi di programmazione ed evitavano la dispersione del lavoro. Questo fu il primo gruppo concreto di condivisione del software. Emacs è nato in questo modo, ma poi è stato ritradotto varie volte.

\subsection{Le prime incursioni}

Allo stesso tempo si iniziarono a vedere le prime debolezze. Il dipartimento di difesa obbligò gli hackers a programmare sui sistemi del dipartimento di \textit{computer science} del MIT, in cui c'era mancanza di accesso libero all'informazione. Arrivarono al punto di fare una sorta di ``sciopero'' nei confronto del laboratorio e bloccarono l'accesso alle succesive versioni di Emacs. Si creò dunque una frattura tra Richard Stallman e la comunità.

Gli hackers cominciarono a migrare e andarono a lavorare altrove per altre società. Comparirono i primi programmi coperti da copyright nel campo dell'\textbf{intelligenza artificiale}.
\clearpage
\section{Il movimento open-source}

\subsection{Materiale di riferimento}

\begin{itemize}

\item \textit{The Daemon, the GNU and the Penguin: a history of Free and Open Source} - Peter Salus;\\
Capitoli 9, 18, 19, 20, 22 \\
Disponibile sotto Creative Common all'indirizzo: \url{http://www.debian.org/doc/manuals/project-history/}
\item \textit{Breve storia di Debian} - disponibile all'url: \url{http://www.debian.org/doc/manuals/project-history/}.

\end{itemize}

\subsection{MINIX}

Unix era un sistema operativo vero e proprio, ed era molto complesso. John Lions, un famoso sviluppatore Australiano, pubblica il codice sorgente commentato di UNIX in un'opera storica denominata: \textbf{Lions' Commentary on UNIX 6th Edition, with Source Code}. Ma con l'avvento della versione 7 di unix vengono imposti tutta una serie di blocchi: l'opera di Lyons viene bloccata, aumentano i costi delle licenze e vi sono delle restrizioni sull'insegnamento in classe. Molte università interruppero dunque l'insegnamento di unix; questo fu un cambiamento abbastanza stupido, perchè ciò che aveva reso forte unix era la diffusione nel mondo accademico. 

Andrew S. Tanenbaum era all'epoca un insegnante di \textit{computer science} e venne molto toccato da questa decisione, in quanto aveva sempre insegnato basandosi su unix. Decise allora di creare \textbf{MINIX}, un sistema operativo abbastanza importante, minimale, per scopi didattici, pensato per essere \textbf{semplice}. Era un sistema a micro kernel, rilasciato sotto \textbf{licenza permissiva} ma non libera. Aveva inoltre scritto un libro che documentava e spiegava MINIX. Ma questo sistema operativo aveva una grossa limitazione: \textbf{mancava un emulatore di terminale}.

\subsection{Linux}

Linus Torvalds è stato uno dei primi utilizzatori di MINIX. Nasce ad Helsinki nel 1969. Nel 1990 frequenta l'Università di Helsinki, comincia a studiare Tanenbaum e comincia a fare le prime modifiche per provare a creare un emulatore di terminale. Nel 1991 Lars Wirzenius (un amico di Torvalds) lo porta alla conferenza di Stallman, nella quale ebbe una prima esposizione al progetto GNU. Il 5 Gennaio 1991 Torvalds:

\begin{itemize}

\item Compra un PC (un 80386);
\item Ci installa MINIX;
\item Inizia a scriverci un emulatore di terminale (scritto in C e in assembly).

\end{itemize}

\begin{figure}[htbp]
\centering
\includegraphics[width=50mm]{images/linus-torvalds.jpg}
\caption{Linus Torvalds}
\end{figure}

La prima versione di \textbf{Linux} è la A e la B, fatta solamente da due finestre. Da emulatore di terminale com'era pensato in origine Linux d lì il è stato espanso fino a crearci un vero e proprio sistema operativo \textbf{Linux 0.0.1} con un kernel funzionante.

Già nel 1992 il sistema era diventato molto importante. Torvalds decide dunque di rendere il sistema \textbf{indipendente} da MINIX (ci fu anche una disputa con Tanenbaum). Cambiò dunque la licenza adottando la GPL, che considerava buona per il suo sistema operativo, a prescindere dal software GNU stesso. Nascono inoltre già le prime distribuzioni basate su linux, come ad esempio SUSE, MCC o la prima distribuzione commerciale: LGX. Queste distribuzioni rendevano decisamente più facile l'utilizzo di Linux (di per sè molto complesso).

Nel 1994 viene rilasciato \textbf{Linux 1.0} e fu lo stesso Torvalds a presentarlo in una conferenza tenutasi ad Helsinki. Già allora era un sistema utilizzabile.

Ancora nel 1993 erano nate le prime versioni commerciali: Bolzern, Flagship e \textbf{Linux Pro}. Nel 1994 nasce inoltre \textbf{RedHat}, creato da Marc Ewing, e diventa ben presto la più diffusa distribuzione Linux.

Nel 1996 fu scelto come logo ufficiale di Linux un pinguino disegnato da Larry Ewing, chiamato \textbf{Tux}, come abbreviazione di \textbf{T}orvalds \textbf{U}nix.

Sempre nel 1996 esce \textbf{Linux 2.0} con supporto a microprocessore. Con la versione 3.0, uscita nel 2011, le modifiche sono state molto più incrementali. Nel corso degli anni con la 2.0 il grado di utenza era ancora molto piccolo.

\begin{table}[htpd]
\centering
	\begin{tabular}[c]{l | l | l}
	\hline
	& 1992 & 1992 \\
	\hline
	Sviluppatori Linux & 100 & 1000 \\
	\hline
	Linee di codice Linux & 250.000 & 14.000.000 \\
	\hline
	\end{tabular}
\caption{Sviluppo di Linux negli anni}
\end{table}

Una volta c'erano molti sviluppatori \textit{volontari}, ad oggi il supporto è dato da grosse aziende che possono investire tempo e denaro su Linux.

\subsection{Debian}

C'era un forte legame tra il mondo degli hackers e il mondo del software libero. Si venne a creare una \textbf{nuova generazione di sviluppatori}. Volevano provare a costruire una distribuzione che fosse fortemente legata a certi principi, che mettesse insieme varie cose, che facesse da collante. A quel punto nacque \textbf{Debian}. Linux da un lato stava procedendo e crescendo velocemente ed aveva molte caratteristiche interessanti, ma dall'altro c'era una lontananza dai principi del software libero e dalla GNU. Questo era percepito come un problema da parte di una fetta della comunità. Ad altri invece la cosa andava più che bene, dunque si venne a creare una \textbf{divergenza}. 

Il progetto Debian venne fondato da Ian Murdock nel 1993, con l'intento di fare una distribuzione di Linux \textbf{completamente libera}. Entrò a far parte del progetto GNU nel 1994-1995. Nel 1994 venne redatto il \textbf{manifesto debian}, nel quale si riassumeva il significato e la filosofia di debian. La prima versione stabile (Debian 1.1 ``Buzz'') venne rilasciata nel 1996. Il project leader della Debian divenne Bruce Perens

La caratteristica principale di Debian è che pone l'attenzione in modo quasi maniacale alla qualità del software (a volte perdendo anche molto tempo) e al fatto che il software sia libero (solo software DFSG. Si basa su una \textbf{forte comunità} che gestisce (tramite votazioni) tutte le decisioni sullo sviluppo; chiunque può proporre cambiamenti e ognuno è responsabile delle proprie azioni (attenzione alla sicurezza). Un altro cardine su cui puntano gli sviluppatori Debian è una strenua \textbf{disponibilità} del software.

Debian è caratterizzato da un suddivisione (politica) in più parti del repository. Le uniche componenti che sono della Debian sono quelle libere e quelle che dipendono da software libero. Il software che è all'interno della Debian entra nell'archivio principale, \textbf{FREE}. Poi all'interno ci sono altre componenti ospitate nel server della Debian ma che non sono della Debian, \textbf{NON-FREE} e \textbf{CONTRIB} (che è libero ma dipende da componenti non libere).

C'è una versione di Debian \textbf{stabile} (software un po' vecchio però), quella che ha superato tutti i bugfix, una versione \textbf{non stabile} e una versione \textbf{testing}. Nella versione non stabile ``\textit{può esplodere tutto da un momento all'altro}'', viene aggiornata ogni giorno. La testing entra nel pacchetto solamente se nell'ultimo mese non sono stati segnalati bug importanti.

Da un lato Debian ha molto software e dall'altro non avendo interessi commerciali, non ha interessi nel penalizzare la concorrenza. È gestita essenzialmente da volontari.

\subsection{La cattedrale e il bazaar}

Un altro personaggio molto importante nel mondo del software libero è Eric Steven Raymond, informatico e programmatore statunitense di grande esperienza, che nel 1997 pubblica ``\textbf{La cattedrale e il bazaar}'', un saggio sullo sviluppo del software. Esso era essenzialmente destinato ad affrontare il problema del perchè Linux, con il suo kernel, funzionasse così bene. Non era scontato che Linux funzionasse, anzi all'inizio, secondo la sua opinione, era destinato a ``scoppiare''. Il problema è che ognuno tendeva a pensare con la propria testa; eppure la cosa non succedeva. Quindi iniziò ad analizzare il fenomeno. 

Una delle prime cose che ha fatto è stato quello di iniziare un progetto, chiamato \textbf{popclient}, per l'invio e la ricezione della posta. Comincia dunque ad utilizzare tutta una serie di principi per la gestione di questo progetto per vedere se riusciva a ricreare il successo che aveva visto con Linux, un progetto che funzionasse bene e avesse una solida base di sviluppatori. Uno dei principi che aveva adottato era di trattare gli utenti come una sorta di \textbf{co-sviluppatori}, come se il programma fosse stato fatto insieme a loro. La comunità popclient divenne dunque molto attiva. 

Il secondo principio su cui basò il proprio lavoro fu: ``Distribuisci presto, distribuisci spesso e presta ascolto agli utenti'', in questo modo si favorisce la risoluzione di bachi in tempi brevi. Molti utenti si facevano avanti e miglioravano il software, sentendosi parte attiva del progetto. L'idea che secondo Raymond aveva fatto il successo di Linux era trasformare il software da uno sviluppo di una persona che ``dona agli altri'' a uno sviluppo ``social'', attorno al quale ruotava una comunità. Spesso la varietà delle persone porta ad una varietà di modi per risolvere il problema, vi sono approcci diversi, e la combinazione dei contributi può significare grossi miglioramenti.

L'articolo di Raymond ebbe una grossissima fortuna, l'impatto all'interno della comunità fu molto sentito, e l'effetto di questo fu che l'attenzione andò oltre la semplice comunità degli appassionati. Una delle conseguenze di questo successo fu che nel 1998 Netscape annunciò di voler rilasciare il codice sorgente del proprio browser, e disse che per decidere questo si era basato anche sull'articolo di Raymond. Una volta Netscape era dominante, prima dell'arrivo del colosso Microsoft con Internet Explorer. Il browser di casa Microsoft è sicuramente disprezzato al giorno d'oggi, ma in realtà per gli standard di allora era molto buono e la Microsoft era riuscita, partendo da zero, a recuperare terreno in tempi notevoli su Netscape. A Netscape non interessava moltissimo del suo navigatore, non poteva infatti guadagnarci (era gratis), ma focalizzava la sua attenzione sul mondo Server. Temeva però che se Internet Explorer fosse divenuto dominante a quel punto i propri server sarebbero stati ``scartati'' in favore di quelli di Microsft. Avrebbe perso una grossa fetta di mercato e buona parte del loro reddito. Ebbero dunque l'idea di rilasciare il codice sorgente di Netscape sotto una licenza libera. In questa decisione aveva avuto un ruolo importante Raymond, che divenne dunque una celebrità.

C'era però il rischio che il termine ``software libero'' si svalutasse o non acquisisse la giusta importanza. Ci fu dunque una riunione su come sfruttare al massimo l'annuncio di Netscape: nasce il termine ``\textbf{open source}''. L'idea era che il software libero garantiva software di maggiore qualità ai fini di offrire una piattaforma stabile, con una buona comunità e aperta. Nel 1998 nasce dunque la \textbf{Open Source Initiative}, fondata da Raymodn e Perens, che aveva come scopo quello di diffondere il mondo open source. In questo movimento entrarono a far parte tutti gli sviluppatori più importante, come RedHat. 

\subsection{La filosofia open source}

Questi principi, che sono delle vere e proprie clausole, sono tutti pragmatici: quello che conta è costruire software che sia affidabile e veloce in modo analogo a Linux:

\begin{itemize}

\item \textbf{Licenze libere e permissive}; quando una licenza è chiusa si tratta gli utenti non come co-sviluppatori ma come ``utenti di serie B''. È come dire: ``\textit{Vabbè, se proprio vuoi fammi il bug reporting...}'' oppure ``\textit{Tu sei talmente inutile per me che nemmeno ti aiuto ad aiutarti...}''. È un principio che in casa Microsoft va bene, perchè l'obiettivo non è costruire una comunità. Nel caso del mondo open-source fare così è come ``darsi la mazza sui piedi'';
\item \textbf{Costruzione di una comunità attorno al software}; il software non è più una cosa che viene \textit{usata}, ma un modo di vivere, una cosa in cui le persone sono coinvolte. Con la crescita e lo sviluppo di una comunità attorno al software non solo i bachi vengono risolti più rapidamente, ma si hanno anche nuovi apporti mentali e contributi da parte delle persone;
\item \textbf{Trasparenza del processo di sviluppo}; la gente deve vedere quello che sta succedendo, perchè una volta che lo vede magari contribuisce. Questo tipo di comunicazione è fondamentale;
\item \textbf{Codice sorgente liberamente disponibile}; altrimenti non possono esserci trasparenza e contributi da parte della comunità;
\item \textbf{Codice sorgente liberamente modificabile};
\item \textbf{Libera redistribuzione}, anche ad uso commerciale;

\end{itemize}

\subsection{Open Source Definition}


\clearpage

\appendix

\section{Glossario}

\subsection*{A}

\underline{\textbf{ARPA}}: %TODO

\underline{\textbf{ARPANET}}: %TODO

\underline{\textbf{Assembler}}: %TODO

\subsection*{B}

\underline{\textbf{Beowulf}}: %TODO

\subsection*{C}

\subsection*{D}

\underline{\textbf{Debian}}: %TODO

\subsection*{E}

\subsection*{F}

\underline{\textbf{Free Software Foundation}}: %TODO

\underline{\textbf{Freeware}}: %TODO

\underline{\textbf{fsf}}: %TODO

\subsection*{G}

\underline{\textbf{GNU}}: %TODO

\underline{\textbf{GPL}}: %TODO

\subsection*{H}

\subsection*{I}

\subsection*{J}

\subsection*{K}

\underline{\textbf{Kernel}}: %TODO

\subsection*{L}

\underline{\textbf{Linux}}: %TODO

\subsection*{M}

\underline{\textbf{Mimix}}: %TODO

\underline{\textbf{MUTIX}}: %TODO

\subsection*{N}

\underline{\textbf{Netscape}}: %TODO

\underline{\textbf{nslu2}}: %TODO

\subsection*{O}

\underline{\textbf{OSI}}: %TODO

\subsection*{P}

\underline{\textbf{PDP}}: %TODO

\subsection*{Q}

\subsection*{R}

\underline{\textbf{RedHat Enterprise Linux}}: %TODO

\underline{\textbf{Routes}}: %TODO



\subsection*{S}

\underline{\textbf{StarOffice}}: %TODO

\underline{\textbf{Sun}}: 

\underline{\textbf{S\&P}}: %TODO

\underline{\textbf{Shareware}}: %TODO

\underline{\textbf{Symbolics}}: %TODO	

\subsection*{T}

\underline{\textbf{TECO}}: %TODO

\subsection*{U}

\underline{\textbf{Unix}}: %TODO

\subsection*{V}

\subsection*{W}

\subsection*{X}

\subsection*{Y}

\subsection*{Z}


	






\end{document}