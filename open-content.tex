\section{Open Content}

\subsection*{Materiale di riferimento}

\begin{itemize}

\item \textbf{Viral Spiral, How the Commoners Built a Digital Republic of Their Own} - David Bollier;
\item La licenza GFDL \url{https://it.wikipedia.org/wiki/GNU_Free_Documentation_License};
\item \textbf{Creative Commons: a user guide} - Simone Aliprandi \url{http://www.aliprandi.org/cc-user-guide/}.


\subsection*{Per approfondire}

\item \textbf{EFF - Wikipedia} \url{https://it.wikipedia.org/wiki/Electronic_Frontier_Foundation}
\item \textbf{EFF Official Website} \url{https://www.eff.org/}
\textit{Il caso Sega vs Accolade} \url{https://en.wikipedia.org/wiki/Sega_v._Accolade}
\item \textbf{Scarichiamoli} \url{ttp://www.scarichiamoli.org}
\end{itemize}

\subsection{I primi tentativi}

La \textbf{GNU Free Documentation License} è partita all'inizio perchè c'era la necessità di avere una \textbf{documentazione libera}. Per molto tempo la documentazione che accompagnava il software libero era sotto licenza GPL e lo stesso TLDP (The Linux Documentation Project) distribuiva la sua documentazione sotto GPL, semplicemente perchè era quello che ``passava al convento'', ma non ha molto senso usare la GPL per della documentazione. Se per esempio i \textit{Promessi Sposi} fosstrasparentiero sotto licenza GPL io potrei prendere i sorgenti e cambiare nella copertina il nome dell'autore: questo posso farlo tranquillamente senza problemi se sono sotto GPL. Ma questo è un problema perchè c'è differenza tra protezione del software e protezione del contenuto testuale. La tutela dell'autore originale è importante. Per questo motivo la GNU Free DOcumentation License si è posta come obiettivo una documentazione che avesse:

\begin{itemize}

\item Libertà di modifica;
\item Tutele dei diritti morali dell'autore;
\item Gestione del problema legato alle copie non trasparenti; come nel caso della GPL si vuole far sì che se qualcuno modifica un prodotto che è sotto GFDL quello rimanga disponibile nei suoi sorgenti (ad esempio non posso redistribuire un documento come serie di immagini jpeg che non sono editabili) e che non limiti la libertà di terzi di modificare la documentazione;
\item Copia in grande quantità e non.

\end{itemize}

\subsection{Concetti fondamentali}

È una licenza che nasce per la \textbf{documentazione} (soprattutto del progetto GNU). Si sente la necessità di dare al prodotto la possibilità di avere una documentazione sempre aggiornata quando esso viene rilasciato. Se ad esempio attuo una modifica al programma è necessario modificare la documentazione che ci sta dietro. 

Le parti del documento da considerare sono:

\begin{itemize}

\item Il \textbf{titolo}; se io modifico un documento sotto GFDL devo mantenere l'autore originale nella copertina aggiungendo eventualmente il mio;
\item Il documento in sè e il suo \textbf{contenuto};
\item \textbf{Sezioni secondarie e invarianti}, come ad esempio le dediche che possono essere modificate solamente mantenendo inalterato il tono. Le sezioni invarianti non possono assolutamente essere modificate ma devono essere al di fuori dell'argomento principale del documento;
\item \textbf{Testi di copertina}; posso modificare la copertina ma devo dichiararmi editore di quella versione;
\item \textbf{Storia del documento}, devo tenere traccia di tutte le modifiche e scriverle in una sezione del documento (esempio sezione Diario delle Modifiche);
\item La \textbf{licenza} dev'essere mantenuta e riportata nel documento.

\end{itemize}

Per quanto riguarda le copie trasparenti ci sono distinzioni per chi distribuisce sotto piccola quantità e grande quantità. In particolare se io voglio pubblicare un documento sul mio sito devo renderlo accessibile. Inoltre non devo aggiungere delle \textbf{restrizioni tecnologiche} che impediscano la modifica. 

Esiste tutta una serie di restrizioni per quanto riguarda la redistribuzione di \textbf{copie senza modifiche}:

\begin{itemize}

\item Mantenimento della licenza;
\item Nessuna misura tecnologica di restrizione;
\item Permettere di esibire la copia in pubblico;
\item Per quanto riguarda le redistribuzioni voluminose:
	\begin{itemize}

	\item Obbligo di identificarsi come autore;
	\item Obbligo di mantenere titoli e testi di copertina;
	\item Obbligo di distribuire una sorgente trasparente.

	\end{itemize}

\end{itemize}

Per quanto riguarda la redistribuzione di \textbf{copie modificate}:

\begin{itemize}

\item Devo modificare il titolo, in questo modo identifico il mio prodotto come qualcosa di diverso;
\item Indicazione degli autori delle modifiche e del documento originale;
\item Rimozione e/o aggiunta dei ``riconoscimenti'';
\item Aggiornamento della sezione cronologia;
\item Preservazione degli invarianti;
\item Preservazione della versione trasparente;
\item Preservazione e/o aggiunta dei testi di copertina.

\end{itemize}

Esistono anche tutta una serie di paragrafi di questa licenza che parlano essenzialmente dell'\textbf{unione di documenti}, delle collezioni e aggregati e delle \textbf{traduzioni} dei documenti. Per quanto riguarda le unioni di più documenti è necessario:

\begin{itemize}

\item Disambiguare tutte le parti;
\item Mantenere una licenza;
\item Rimuovere i riconoscimenti.

\end{itemize}

Un caso particolare è quando si tratta delle traduzioni: la traduzione è comunque considerata una modifica e in questo caso è necessario conservare la licenza e allo stesso tempo conservare gli invarianti. In questo modo si da garanzia all'autore che la sua opera non possa venire mal interpretata perchè io potrei tradurla in maniera non appropriata (scritta male o con ambiguità). È possibile in ogni caso prendere accordi con l'autore.

\subsection{Il fallimento del copyright}

C'è stato un cambiamento enorme per quanto riguarda i tipi di diritti associati agli utenti e agli sviluppatori del software. 

\subsection{Gli anni d'oro della proprietà intellettuale}
\subsubsection{Il caso Nation - Ford}

Il presidente Gerald Ford scrisse una biografia includendo un anedotto sulla sua decisione di perdonare Richard Nixon. 
Ford aveva concesso sotto licenza i suoi diritti di pubblicazione alla Harper\&Row con un contratto per pubblicare un estratto delle memorie sulmagazine Time. Invece, la rivista "The Nation" pubblico dalle 300 alle 400 citazioni letterali direttamente dal libro di 500 pagine senza il permesso ne di Ford, ne della Harper\&Row ne del Time.
Vista questa pubblicazione anticipata, il Time si tirò indietro dal contratto (il che era permesso da una clausola) e la Harper\&Row intentò una causa legale contro The Nation.

The Nation affermò che essendo Ford un personaggio pubblico e che le informazioni da lui divulgate fossero di vitale importanza la pubblicazione ricadesse sotto il fair use.

Un primo verdetto diede ragione alla Harper\&Row, ma uno successivo
della corte d'appello ribaltò la sentenza a favore di The Nation. Harper\&Row ricorse alla corte suprema la quale ribaltò di nuovo il
verdetto stabilendo che il fair use non è una difesa in caso di  
pre-pubblicazione e non permette l'appropriazione commerciale del 
lavoro di un famoso personaggio politico semplicemente 
a causa del pubblico interesse nell'apprendere informazioni su una
figura politica di un evento storico.

\url{https://en.wikipedia.org/wiki/Harper_%26_Row_v._Nation_Enterprises}

\subsubsection{I ruoli di Kastenmeier e Schrader}

Robert Kastenmeier fu un politico americano membro del partito democratico. 
Dorothy Schrader fu consigliere generale per il Copyright Office degli Stati Uniti entrambi diedero un significativo contributo al Copyright Act del 1976.

\subsection{I prodromi della crisi}

Con l'avanzare di internet negli anni '90, il boom delle vendite di apparecchi elettronici e i film su videocassette si ha una contrapposizione tra utenti e lobby pro proprietà intellettuale che si fanno ancora più aggressive, arrivando a brevettare pressoché tutto (persino forme di vita come semi per l'agricoltura).

Dal punto di vista del software nello stesso periodo nasce l'EULA \left(\textit{End-User Licence Agreement}\right), un contratto tra il fornitore di un programma software e l'utente finale. Tale contratto assegna la licenza d'uso del programma all'utente nei termini stabiliti dal contratto stesso e nel 1990 nasce EFF \left(\textit{Electronic Frontier Foundation}\right) un'organizzazione internazionale non profit di avvocati e legali rivolta alla tutela dei diritti digitali e della libertà di parola nel contesto dell'odierna era digitale.

Tornando ai brevetti, nel 1999 amazon arrivò a brevettare negli USA gli acquisti One-click (brevetto rifiutato nell'Unione Europea a causa di un mancato processo inventivo).

Negli anni '90 fece molto discutere il caso \textit{SEGA - Accolade}. Accolade casa produttrice e rivenditore di videogiochi, disassemblò tramite tecniche di Reverse Engineering alcuni giochi per la console Genesis di SEGA, in modo da poter pubblicare giochi compatibili per la console di SEGA senza il loro consenso. Il caso fece molto parlare di se visto che riguardava vari aspetti controversi come il copyright, l'uso permissivo di marchi e il fair use applicato al codice.
Nel 1992 SEGA vinse e Accolade fu costretta ad interrompere la produzione di giochi compatibili e a ritirare le copie dal mercato, Accolade ricorse in appello e il 28 agosto 1992 vinse (anche a seguito di forti pressioni da varie associazioni e del clamore mediatico del caso), quando la corte d'appello stabilì che la decompilazione ricadeva nel fair-use e condannò per altri motivi SEGA al pagamento di tutte le spese processuali. Alla fine le due compagnie trovarono un accordo conveniente ad entrambe. Il caso servì a stabilire che i principi funzionali del software non possono essere protetti da diritti d'autore (non e' possibile brevettare un algoritmo) e la corte incoraggiò la copia di software protetto per l'esplorazione di funzionalità non protette.

Altro evento importante fu l'approvazione da parte dell'amministrazione Clinton del \textit{National Information Infrastructure} un insieme di regole atte a costruire reti di comunicazione, servizi interattivi, hardware e software interattivo, per rendere disponibile sia al settore pubblico che privato una marea di informazione senza favorire un'azienda sull'altra. Purtroppo vennero anche introdotte ulteriori restrizioni come la rimozione dei diritti di prima vendita e il DRM.

\subsection{Il movimento degli accademici}

\subsection{Mickey Mouse Protection Art}

\subsection{Eldred vs Reno}

\subsection{Creative Common}

\subsection{Le licenze Creative Common}