\section{Storia del software libero}

\subsection{Gli albori}

1950 - 1960, la cultura hacker nasce ai laboratori del \textbf{MIT}. Nel 58/59 nascono i primi corsi di \textbf{intelligenza artificiale} (di fatto era informatica). Vi era un rapporto \textit{giocoso} tra il gruppo di ricerca e i ragazzi, un rapporto che funzionava molto bene. Il primo gruppo hacker nasce come sottogruppo di \glossario{S\&P}. All'inizio non era una vera e propria comunità hacker ma c'era solamente un gruppo che frequentava i corsi e morta lì. L'accesso ai PC all'epoca era molto riservato (ai docenti, al personale). Questo gruppo aveva trovato una sala (407) con delle macchine perforatrici (per programmare le schede perforate). 

Il rapporto cambierà con l'evoluzione delle tecnologie, con l'accesso più libero alle nuove macchine (ATX0). Con il cambiamento delle tecnologie cambia anche la gestione (si poteva accedere liberamente alle macchine), si cominciò a formare un gruppo di persone che \textit{bazzicavano} sulle macchine. Si formò così una comunità con idee e un'etica in comune. 

Con il \glossario{PDP} (computer) le cose cambiarono ulteriormente. IL PDP era pensato non per il \textit{best-processing} ma per una computazione interattiva (aveva un monitor ...), costava inoltre molto meno ed era quindi più accessibile. A quel punto diventava possibile utilizzare le macchine a costo zero.

\subsection{Etica hacker}

L'etica che si sviluppò all'interno della comunità si basava sui seguenti 4 punti:

\begin{enumerate}

\item \textbf{Libero accesso all'informazione}, bisogna poter metterci le mani, e questo non era all'epoca garantito a tutti;
\item \textbf{Decentralizzazione}, potere che si sposta dal centro;
\item Gli hacker dovrebbero essere \textbf{giudicati solo per il loro valore};
\item Il software come \textbf{espressione artistica}, va oltre quello che è la pura utilità, va a fare qualcosa che è divertente e bello (creazione di giochi). Doveva essere un piacere che andasse agli altri;

\end{enumerate}

\subsection{Conseguenze dello sviluppo software}

In questo momento c'è una libera condivisione del codice, non esiste software protetto da copyright. Non c'era interesse nel proteggerlo. Il software era un'appendice dell'hardware. Le macchine erano in continua evoluzione e modifica. L'avere un accesso a come il software funzionava era vitale per gli sviluppatori. Dato che il tempo macchina era oneroso, bisognava sfruttarlo al meglio e quindi era importante la condivisione del lavoro tra gli sviluppatori.

\textbf{Incopatible Time Sharing}, ci si compra un terminale e poi si accede alle reti di calcolo di un enorme computer. ITS era la base di questo concetto. Non c'erano password, ogni utente aveva i propri file personali accedibili da tutti. Era disegnato per la cooperazione tra gli hacker ed assumeva una grossa fiducia negli utenti.

\subsection{La fine del periodo degli hacker}

\begin{itemize}

\item 68: il computer lab viene isolato;
\item 70's-80's: frammentazione della cultura hacker:

	\begin{itemize}

	\item 1976: copyright act;
	\item \glossario{Symbolics} recluta la maggior parte degli hacker rimasti al MIT;
	\item Altri hacker fondano \glossario{LMI}.

	\end{itemize}

\item Negli '80 cessa la produzione del PDP-10, condannando definitivamente ITS.

\end{itemize}

\subsection{Nasce unix}

Una serie di circostanze ha voluto che \glossario{unix} venisse rilasciato con licenza libera (era comunque protetto da copyright). Nasce come una risposta a \glossario{MUTIX} per creare un sistema diverso, più piccolo e con minor risorse. Fu il primo sistema scritto non direttamente in linguaggio macchina ma scritto in linguaggio C. Questo sistema ebbe subito un grosso successo anche e soprattutto in ambito universitario. 

L'\glossario{AT\&T} era un monopolio telefonico del governo, aveva la restrizione che non poteva vendere software e quindi si decise di distribuire gratuitamente unix. Al suo lancio esso si diffuse molto rapidamente. Una delle università che lo utilizzò fin da subito fu \textbf{Berkeley}, che aveva interesse dal punto di vista della programmazione interna di unix. Nel 1977 nacque \glossario{BSD}, una versione modificata di unix utilizzata dall'università. Due anni dopo la AT\&T annunciò di voler commercializzare unix. 

Nel 1983 AT\&T viene divisa e unix diventa commerciale. A questo punto c'era BSD ed era una possibile scelta. \glossario{ARPA} lo utilizzò per creare \glossario{ARPANET}, per collegare diversi computer via network, garantendo un'uniformità a livello software del sistema operativo. Il software libero ebbe quindi un impatto notevole. ARPA mise inoltre a disposizione dei fondi per migliorare BSD e facilitare il suo sviluppo. In seguito venne citata in giudizio dalla AT\&T perchè si diceva volesse il codice di unix. Questo bloccò di fatto lo sviluppo di ARPANET. Se ciò non fosse successo probabilmente non sarebbe nato unix.

\subsection{Nasce il progetto GNU}

