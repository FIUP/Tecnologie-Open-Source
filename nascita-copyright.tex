\section{La nascita del copyright}

\subsection*{Materiale di riferimento}

\begin{itemize}

\item \textit{Privilege and Property Essays on the History of Copyright} - Ronan Deazley, OpenBook Publishers;
\item \url{http://digital-law-online.info/lpdi1.0/index.html} - sezione riguardo il copyright sul software;

\end{itemize}

\subsection{La storia}

Le prime forme di protezione in realtà non erano pensate per proteggere i diritti del beneficiario ma per dare dei vantaggi alle autorità che le emanavano; in secondo luogo non erano collegate alla conoscenza che raccoglieva quello si voleva proteggere, non si proteggeva il contenuto del libro ma si proteggeva l'ente industriale che lo aveva prodotto. Infine non erano collegate nemmeno all'autore. La forma era molto diversa da quella odierna.

Il copyright si è sviluppato in origine a Venezia intorno al 1469, 13 anni dopo la produzione della bibbia di Gutemberg. Prima dell'invenzione della stampa non c'era un sistema ben strutturato e organizzato, il libri costavano moltissimo, richiedeva anni di lavoro e di conseguenze non ve ne erano molti. Era un processo molto impegnativo la scrittura di un libro, stimato intorno agli 80.000 \euro{} di oggi. Nel 1450 la bibbia di Gutemberg cambia decisamente le carte in gioco, viene creato un processo industriale della scrittura, che prima era quasi un'``opera d'arte''. Lo stato incominciò ad interessarsene, per controllare il flusso di informazioni ed imporre dei blocchi sulla conoscenza; questa fu la direzione presa in Inghilterra. Dall'altro lato la stampa era un'invenzione fenomenale e si voleva trarre vantaggi da essa; questa fu la direzione presa a Venezia, i quali erano molto interessati ad avere il sistema di stampa e a sfruttarlo; cercarono di fare in modo che tanta gente ce l'avesse, imponendo comunque dei controlli. In quest'ottica il copyright non nasce come un diritto, ma come forma di privilegio che l'autorità concede, ha una forma di incentivo brevettuale.

In Italia esistevano delle \textbf{corporazioni}, che detenevano il controllo sulla conoscenza delle arti artigiane, vi era tutto un sistema di privilegi ed erano loro a mantenere l'ordine. Dall'altra parte gli stessi comuni che avevano creato queste corporazioni avevano anche creato un sistema per incentivare la gente degli altri comuni di svelare la conoscenza e diffondere le tecniche più avanzate. 

Quando nel 1469 Johannes of Speyer andò a Venezia chiese una forma di incentivo per portare la propria macchina di stampa a Venezia, e ovviamente glielo concessero, perchè era una macchina importante dalle grandi potenzialità. Gli diedero dunque un'esclusiva sulla stampa per 5 anni. Al giorno d'oggi quello che conta di un libro è il suo contenuto, non la forma; all'epoca si pagavano i libri in funzione del peso o come merce di scambio. Pochi mesi dopo questa esclusiva però Johannes morì, e questo privilegio durò dunque per pochi mesi, e preso si cominciò dunque a formare un mercato sulla stampa. 

Una cosa importante che fu conseguenza di questo privilegio fu che il controllo della stampa venne sottratto alle corporazioni, la produzione si orientò dunque in un certo modo, non ci fu la differenziazione del modo in cui venivano gestiti i privilegi di stampa. Questi privilegi erano concessi volta per volta alle singole persone ed erano associati al modo in cui venivano prodotti i libri:

\begin{itemize}

\item Privilegi e non diritti d'autore; 
\item Era lo stampatore ad avere i diritti;
\item Carattere tecnologiche dei privilegi iniziali, non associati al contenuto.

\end{itemize}

Le opere all'epoca erano per la maggior parte diverse edizioni delle opere classiche.

Uno \textbf{statuto} importantissimo, quello del 1474, per la prima volta stabiliva che quando una persona produceva qualche cosa di meritevole, di nuovo e originale, aveva diritto ad una protezione per 10 anni. Questa sembra per la prima volta una forma di protezione collegata alla proprietà intellettuale di quello che ci sta dentro e non esclusivamente ad un processo industriale. Era una cosa che si avvicinava a un \textit{diritto}. Ma di fatto questo statuto finì in un binario morto ma ebbe un effetto molto importante, perchè si spostò l'attenzione per la prima volta dall'interesse degli stampatori agli \textbf{autori}. 

Il 1517 segna un cambiamento nel modo con cui vengono distribuiti i privilegi. Prima i privilegi venivano distribuiti in modo indipendente dal contenuto, quello che interessava era esclusivamente il processo di stampa. A un certo punto quando ci si stanca di avere tutti libri uguali della stessa opera, volevano avere libri un po' più \textit{nuovi}, e quindi ritirarono tutti i privilegi sui libri in stampa e le opere cadono nel pubblico dominio. Fu necessario dunque lo spostamento del mercato verso le opere originali, le quali erano proteggibili. Per una volta quello che conta non è il modo in cui viene stampato un libro ma quello che ci sta dentro. Gli \textbf{autori} cominciarono ad avere dunque un po' più di potere. Questa protezione sulle opere si rafforzò nel tempo, andando a proteggere anche le \textbf{modifiche} sulle opere. Come conseguenza di questo i privilegi cominciarono ad essere garantiti anche agli autori.  

La regolamentazione delle arti artigiane era fatta dalle corporazioni, ma piano piano sempre più persone le stavano trovando più adatte. Si stava entrando nel rinascimento, un'epoca in cui si da più spazio all'uomo e alla sua creatività. Le corporazioni avevano il compito di regolamentare le arti, di proteggerle; ogni comune aveva le proprie. Una conseguenza importante di ciò fu la nascita del concetto di ``\textbf{proprietà immateriale}''. Si aveva la netta sensazione che quello che una persona conosceva era importante. D'altra parte la forma di protezione era strettamente legata alla comunità e la comunità va protetta proteggendo l'informazione. Questa forma di privilegio era molto legata agli autori e non più ai produttori. Si spostò l'interesse dal processo di produzione del libro al suo contenuto e in particolare all'autore.

Il sistema brevettuale parallelamente collegò il concetto di proprietà immateriale alla persona, anche perchè questo è un periodo in cui c'è uno spostamento generale dalla comunità alla persona (umanesimo). Questo ebbe un grosso impatto. Prima chi gestiva la conoscenza erano degli artigiani, con la nascita di un interesse culturale diventa più ``teorico'', nasce una differenza tra la proprietà intellettuale e i suoi prodotti. Questo venne rafforzato ulteriormente dalla nascita degli \textbf{scrittori di professioni}. Il valore delle opere deriva dall'individuo e dalle sue conoscenze.

Nel 1600 in Inghilterra si era sviluppata tutta una vita politica pubblica, in cui si discuteva pubblicamente e ci si formava un'opinione, per esempio nei vari caffè. A un certo punto si sentì il bisogno di controllare questa opinione pubblica; prima esisteva una struttura chiamata \textit{camera stellata} che era un tribunale ``fittizio'', una forma di censura che permetteva di controllare ciò che veniva espresso dalla gente. Nel 1641 venne abolita e a quel punto ci fu la necessità di sostituire questa forma censoria. Questi controlli vengono implementati nel 1643/1644, in ci ci fu un rigido controllo pre-stampa e la censura fu affidata alla \textbf{stationer's company}, alla quale era affidato il compito di decidere chi poteva stampare. Naturalmente il diritto lo aveva solo chi si dimostrava premuroso nei confronti dei diritti della corona. Questa legge venne prorogata diverse volte fino al 1695. 