\section{L'ultimo degli hacker}

Analizziamo la fase di decadenza degli hacker, qui entra in gioco Richard Stallman.

\subsection{Prime esperienze}

Nasce a New York dove si forma dal punto di vista informatico, diventa un esperto di \glossario{Assembler}, di sistemi operativi e di editor di testo. Poi entra ad Harvard e si laurea in Fisica. Scopre in segreto un'affinità con la cultura hacker sviluppatasi al MIT. Venne assunto al MIT come programmatore di sistemi. Viene preso sotto l'ala protettiva di Richard Greenblatt e Bill Gosper, che gli fanno da mentore.

\subsection{Emacs}

Una delle prime cose che Stallman fece fu la creazione di \glossario{Emacs}. Inizialmente esso era un insieme di macro per \glossario{TECO}, che era una sorta di linguaggio per scrivere e non era assolutamente real-time. Creò dunque una sorta di editor dentro TECO. Guy Steele ebbe in seguito l'idea di fare un po' di ordine tra le macro (che erano diventate tantissime). Quest'opera fu continuata da Stallman, in modo da avere un insieme più omogeneo. Questo disordine non sarebbe dovuto replicarsi in futuro, bisognava evitare che ciascuno costruisse il proprio insieme di macro. Quindi si imposero delle restrizioni sulla redistribuzione delle macro. Questa condizione creò una sorta di comunità in cui i programmatori condividevano gli sforzi di programmazione ed evitavano la dispersione del lavoro. Questo fu il primo gruppo concreto di condivisione del software. Emacs è nato in questo modo, ma poi è stato ritradotto varie volte.

\subsection{Le prime incursioni}

Allo stesso tempo si iniziarono a vedere le prime debolezze. Il dipartimento di difesa obbligò gli hackers a programmare sui sistemi del dipartimento di \textit{computer science} del MIT, in cui c'era mancanza di accesso libero all'informazione. Arrivarono al punto di fare una sorta di ``sciopero'' nei confronto del laboratorio e bloccarono l'accesso alle succesive versioni di Emacs. Si creò dunque una frattura tra Richard Stallman e la comunità.

Gli hackers cominciarono a migrare e andarono a lavorare altrove per altre società. Comparirono i primi programmi coperti da copyright nel campo dell'\textbf{intelligenza artificiale}.